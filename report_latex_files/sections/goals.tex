\section*{Goals}

There are many goals for this project. They are as follows:

\begin{itemize}
    \item \textbf{\textit{Interactivity / Interactively Visualize and Compare} }: Often the case, the result of simulations is static in the sense that if certain output values need to be generated, the corresponding script or program has to be re-run. In other words, usually there is no dynamic adjustments that can be made to the program when it is running. However, the application being designed in this project achieves this in the simplest and powerful ways through a simple slider. The user can adjust the sliders and watch the output graph change as a result of the configurations. This is how we achieve interactivity through the user. In addition, this behavior enables the user to also ask complex questions such as how does one simulation result compare or change with the other graph.
    
    \item \textbf{\textit{Flexibility / Process data from any location} }: In most cases, if a user wants to execute a script with a certain input file, the user has to download the input data file and run the scripts on the local downloaded data. However, this approach is not very efficient. Our application plans to eliminate this behavior by adding flexibility to the input sources. In other words, the data which is input into the program can be a local file or a remote file located on a server. From this behavior, any user can simply point the application to the input source and the entire application efficiently processes the input without any issues. 
    
    \item \textbf{\textit{Scalability/ Compare one or two simulations on the same resultant graph} }: Even if the application is interactive and flexible, if a user can only view the result of only one simulation at a time, then the approach can be very inefficient. In addition, as was previously explained, the user might want to answer interesting questions through the comparison of two different simulations simultaneously. This application is scalable in a sense that it can visualize one or two graphs together, thereby improving the efficiency and simplicity. 
    
    \item \textbf{\textit{Portability/ Runs on any platform} }: \textbf{Portability is probably the most important goal in this project.} In order to create a powerful and a simple application at the same time, the user has to be able to download and run on any platform with the minimum amount of libraries required. In other words, the number of dependencies should be minimum (or if not any) and the application must be simplistic to execute. The entire structure of this application was designed based on these needs and therefore achieves this goal in the most effective manner possible.   
\end{itemize}

