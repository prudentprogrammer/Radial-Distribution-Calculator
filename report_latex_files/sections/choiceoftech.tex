\subsection*{Evaluation}

\begin{itemize}
    \item Programming Languages: Generally in the area of molecular computing or scientific computing in general, speed is of great concern since the simulation should run in a relatively short period of time. Due to this, languages such as C/C++ or MATLAB are often used since they efficiently compute the intermediate results. However, the disadvantage of these languages is that they are not portable. This means that if the application is to be run, then the program has to recompiled on the architecture to be run on and then executed. So in short, this means that code is precompiled and run in efficient manner. Another con of languages such as C/C++ is that higher level functions such as parsing XML files, crawling the web, going through files in a filesystem are more involved and harder to write when compared to interpreted or scripting languages such as Python or R. In these scripting languages, the higher level functions described above can be easily performed with clean and elegant code. However, the speed in these languages often suffer. The interpreted code is way slower than compiled code on most cases and it truly makes a difference in the scientific computing applications. However, the greatest advantage of languages such as python is that the code is portable and can be run on any platform without precompiling.
    
    \item Visualization:
    
    
\end{itemize}


\subsection*{Implementation}