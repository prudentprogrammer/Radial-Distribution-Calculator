Running the following application is simple, the user has to first edit a configuration file which is named \verb|gofr_config.py|. In this configuration file, the user specifies the first input source and second input source. If the second input source is not applicable, the second field is left empty. It should be noted that the URL can be a local file or a file on a server. The user can also specify whether all the configs needed to be taken into account or just the first atomset. The last 5 variables are the important parameters set by the user which is used in the gofr program. These variables represent important quantities such as the atoms being compared and the stepsize to be recorded.



\begin{lstlisting}[language=Python, caption=Config file for setting parameters during the run.]
# Configuration file

first_input_source = "./input_files/md120_short.r"
second_input_source = "./input_files/md120_short2.r"
all_or_first = "a"

first_molecule_name = "H"
second_molecule_name = "H"
rmax = 5
dr = 0.05
stepsize = 1
\end{lstlisting}





After this step is performed, the user can then run the application via python. In this case, the user types ``python run_gofr.py''. This program reads the configuration parameters from the file, extracts the data from the input xml source(s), and runs the gofr algorithm behind the scenes. As a result an output text file called \verb|cum_counts.txt| is outputted for the visualization phase. This file represents the cumulative counts recorded at every $nth$ stepsize. 


\begin{lstlisting}[language=bash, caption=Running the script.]
[~/Documents/Radial-Distribution-Calculator]$ python run_gofr.py
....
....
\end{lstlisting}

After this phase, the gofr algorithm is run on the given data and the intermediate values are present to be graphed during the later stages. The second step is as easy as the first and merely involves running the \verb|visualizer.py| script. This script is run very similar to the run_gofr script, however the user can specify certain options such as x and y limits if neeeded. For instance, ``python visualizer.py cum_counts.txt -x 5 -y 5'' will generate a html page with x and y bounds of 5 and 5. If the user does not specify x and y, then the visualizer automatically sets the bounds for the diagram.

\begin{lstlisting}[language=bash, caption=Running the script.]
[~/Documents/Radial-Distribution-Calculator]$ python Gofr_class.py
....
....
\end{lstlisting}



\begin{lstlisting}[language=bash, caption=Running the script.]
[~/Documents/Radial-Distribution-Calculator]$ ./visualizer.py intermediate_files/cum_counts.txt intermediate_files/cum_counts2.txt 
\end{lstlisting}


\subsection*{Interactivity}

A slider is present on the top of the webpage. Using this feature, the user can select the appropriate range (starting frame number - ending frame number) and effectively analyze the configuration. When the user releases the sliders, the graph changes accordingly. Below is a diagram of what the slider looks like:




\includegraphics[scale=0.25]{images/slider}



Below are two examples of applications run using the following scripts. The first demonstration involves analyzing water and the following involves analyzing silicon. 



Examples of water with two short simulations.

\includegraphics[scale=0.25]{images/two_graphs_HH}
\includegraphics[scale=0.25]{images/two_graphs_OH}
\includegraphics[scale=0.25]{images/two_graphs_OO}



Examples of water with one large simulation (only H vs H shown here)

%\includegraphics[scale=0.25]{images/one_graph_HH}



Explanation of silicon (molecules being compared are Si vs Si).


%\includegraphics[scale=0.25]{images/one_graph_SiSi}
