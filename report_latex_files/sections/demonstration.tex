\section*{Demonstration}

In order to run the application, the user has to first edit a configuration file named \verb|gofr_config.py|. In this configuration file, the user specifies the first input source and second input source. If the second input source is not applicable, the second field is left empty. It should be noted that the input source URL can be a local file or a file on a remote server. This allows for easy comparison of data generated by different research groups. The user can also specify whether all the time steps in the data need to be considered or just the first time step. The last 5 variables represent other important quantities such as the species of the atoms being compared and the stepsize used to save the data. An example configuration file (before the application is run, might look like this):

\singlespacing
\begin{minted}{python}
# Configuration file
first_input_source = "./input_files/md120_short.r"
second_input_source = "./input_files/md120_short2.r"
all_or_first = "a" # a means all and f means first

first_molecule_name = "H"
second_molecule_name = "H"
rmax = 5
dr = 0.05
stepsize = 1
\end{minted}
\doublespacing


After this step is performed, the user can run the application via python. In this case, the user types ``python run\_gofr.py'' in the terminal or command prompt. This program reads the configuration parameters from the file, extracts the data from the input XML source(s), and runs the calculation of the radial distribution function. As a result an output text file called \verb|cum_counts.txt| and a second file (if applicable), \verb|cum_counts2.txt| is generated for the visualization phase. This file contains the cumulative counts recorded at every $nth$ step. A usage of the script is shown below:


\begin{minted}{bash}
[~/Documents/Radial-Distribution-Calculator]$ python run_gofr.py
....
....
\end{minted}

After this phase, the gofr algorithm is run on the given data and the intermediate values are recorded for visualization. The second step is as easy as the first and merely involves running the \verb|visualizer.py| script. This script is executed in a very similar manner to the run\_gofr.py script, however the user can specify certain options such as x and y limits if needed. For instance, \verb| python visualizer.py cum_counts.txt -x 5 -y 5| will generate a output html page with $x$ and $y$ bounds of $5$ and $5$. If the user does not specify $x$ and $y$, then the visualizer automatically sets the bounds for the diagram. A usage of the script is shown below:

\begin{minted}{bash}
[~]$ ./visualizer.py intr_files/cum_counts.txt intr_files/cum_counts2.txt 
\end{minted}


\subsection*{Interactivity}

A slider is present on the top of the output html web page. Using this feature, the user can select a sub-interval of the simulation by choosing the appropriate range (starting step number and the ending step number). When the user releases the sliders, the graph changes accordingly. The following diagram shows the slider:

\begin{figure}[H]
\centering
\includegraphics[scale=0.50]{images/slider}
\end{figure}

We demonstrate the use of the application on two different simulations. The first demonstration involves two short molecular dynamics simulations of liquid water (20 steps) at a temperature of 400 K, and the following involves a molecular dynamics simulation of liquid silicon at a temperature of 2000 K (~1700 C). 

In the case of the water simulation, the application is used for three combinations of choices of atomic species: OO, OH and HH, where O denotes oxygen and H hydrogen. The simulation data is taken from the PBE400 dataset, published at http://www.quantum-simulation.org. 

\begin{figure}[H]
\centering
\includegraphics[scale=0.30]{images/two_graphs_HH}
\caption{Hydrogen-hydrogen radial distribution function $g_{HH}(r)$ for two different simulations of liquid water.}
\end{figure}

\begin{figure}[H]
\centering
\includegraphics[scale=0.30]{images/two_graphs_OH}
\caption{Oxygen-hydrogen pair distribution function $g_{OH}(r)$ for two different simulations of liquid water.}
\end{figure}

\begin{figure}[H]
\centering
\includegraphics[scale=0.30]{images/two_graphs_OO}
\caption{Oxygen-oxygen pair distribution function $g_{OO}(r)$ for two different simulations of liquid water.}
\end{figure}

% \begin{figure}[h!]

% \end{figure}

The following example demonstrates the use of the application for a long simulation (2000 steps) of liquid water, showing that the pair distribution functions become smoother 

\begin{figure}[H]
\centering
\includegraphics[scale=0.30]{images/one_graph_HH}
\caption{Hydrogen-hydrogen radial distribution function $g_{HH}(r)$ for a long simulation of liquid water.}
\end{figure}

\begin{figure}[H]
\centering
\includegraphics[scale=0.30]{images/one_graph_OH}
\caption{Oxygen-hydrogen radial distribution function $g_{OH}(r)$ for a long simulation of liquid water.}
\end{figure}


\begin{figure}[H]
\centering
\includegraphics[scale=0.30]{images/one_graph_OO}
\caption{Oxygen-oxygen radial distribution function $g_{OO}(r)$ for a long simulation of liquid water.}
\end{figure}

The following figure shows the use of the application for a simulation of liquid silicon at 2000 K. The simulation includes 10000 time steps.

\begin{figure}[H]
\centering
\includegraphics[scale=0.30]{images/one_graph_SiSi}
\caption{Silicon-silicon radial distribution function $g_{SiSi}(r)$ for a simulation of liquid silicon.}
\end{figure}

As we can see, the application is scalable and produces the correct results consistently. The first example had $20$ time steps, the second $2,000$, and the third $10,000$ and the visualization application worked as intended This leads to a powerful and a scalable solution.
