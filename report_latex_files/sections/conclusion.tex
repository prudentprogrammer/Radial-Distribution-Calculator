The main goal of this project was to build a portable visualization tool to analyze atoms and their corresponding radial distribution functions in various substances through interactive means. This goal was substantiated by running the application on the water and silicon dataset (as described in the demonstration section). In addition, portability and interactivity was achieved through the user of Python and Javascript. More specifically, Python was used as a means of running the radial distribution function on any platform. Once this was performed, another Python script generated a html page containing the Javascript code to effectively visualize the atomset and create an interactive interface. The user could then use the sliders on the webpage to dynamically analyze the behavior of the atoms and could export the desired graphs. Thus, a very simple and a powerful application was designed in the course of this project to be used by any researcher on any platform desired.