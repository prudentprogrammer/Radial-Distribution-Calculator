\section*{Appendix}

\begin{minted}{python}

'''
Filename: run_gofr.py
'''
import unitcell
import sys
from math import sqrt, pi, pow
from qbox_xyz import QBox_XYZ
import gofr_config as gc
from gofr import Gofr

if __name__ == "__main__":
  if gc.first_input_source != '':
    x1 = QBox_XYZ(gc.all_or_first, gc.first_input_source)
    output = x1.process()
    output =  (' '.join(output))
    output = output.split("\n")
    output = [x.strip() for x in output]
    output = [x for x in output if len(x) != 0]
    gofr_obj = Gofr(output, gc.first_molecule_name, gc.second_molecule_name, 
    gc.rmax, gc.dr, "intermediate_files/cum_counts.txt")
    gofr_obj.process()
    print 'Done processing first input source...'
  
  
  if gc.second_input_source != '':
    x2 = QBox_XYZ(gc.all_or_first, gc.second_input_source)
    output = x2.process()
    output =  (' '.join(output))
    output = output.split("\n")
    output = [x.strip() for x in output]
    output = [x for x in output if len(x) != 0]
    gofr_obj = Gofr(output, gc.first_molecule_name, gc.second_molecule_name, 
    gc.rmax, gc.dr, "intermediate_files/cum_counts2.txt")
    gofr_obj.process()
    print 'Done processing second input source...'

##############################################################
'''
Filename: unitcell.py
'''
'''
Filename: unitcell.py
Author: Arjun Bharadwaj (translated from C++ code written by Dr.Gygi)
Description: This file represents a 3-d molecule in space.
'''

import math
from pprint import pprint

class UnitCell(object):

  '''
  Class constructor
  Arguments: a0, a1, a2 represent the three x, y, and z vectors in space
  The constructor initializes the vectors and more importantly calculates 
  important quantities such as volume and matrix computations representing
  molecular computations.
  ''' 
  def __init__(self, a0=[0.0,0.0,0.0], a1=[0.0, 0.0, 0.0], a2=[0.0,0.0,0.0]):
    
    allVectors = [a0, a1, a2]
    self.aMatrix = a0 + a1 + a2
    
    # volume = det(A)
    self.volume = self.scalarProduct (a0, self.crossProduct(a1, a2))
    
    if self.volume > 0.0:
      fac = 1.0 / self.volume
      amt0 = [fac * elem for elem in self.crossProduct(a1, a2)]
      amt1 = [fac * elem for elem in self.crossProduct(a2, a0)]
      amt2 = [fac * elem for elem in self.crossProduct(a0, a1)]
      
      aMatrixInverse = []
      for e1, e2, e3 in zip(amt0, amt1, amt2):
        aMatrixInverse.append(e1)
        aMatrixInverse.append(e2)
        aMatrixInverse.append(e3)
        
      aMatrixInvTranspose = amt0 + amt1 + amt2
      
      self.bVec0 = [2.0 * math.pi * elem for elem in amt0]
      self.bVec1 = [2.0 * math.pi * elem for elem in amt1]
      self.bVec2 = [2.0 * math.pi * elem for elem in amt2]
      
      self.bMatrix = self.bVec0 + self.bVec1 + self.bVec2
    else:
      # Initialize everything to zeros
      self.bVec0 = [0.0] * 3
      self.bVec1 = [0.0] * 3
      self.bVec2 = [0.0] * 3
      self.aMatrixInverse = [0.0] * 9
      self.bMatrix = [0.0] * 9
    
    bVectors = [self.bVec0, self.bVec1, self.bVec2]
    
    self.anMatrix = []
    self.anMatrix.append(allVectors[0])
    self.anMatrix.append(allVectors[1])
    self.anMatrix.append(allVectors[2])
    self.anMatrix.append([x + y for x, y in zip(allVectors[0], allVectors[1])])
    self.anMatrix.append([x - y for x, y in zip(allVectors[0], allVectors[1])])
    self.anMatrix.append([x + y for x, y in zip(allVectors[1], allVectors[2])])
    self.anMatrix.append([x - y for x, y in zip(allVectors[1], allVectors[2])])
    self.anMatrix.append([x + y for x, y in zip(allVectors[2], allVectors[0])])
    self.anMatrix.append([x - y for x, y in zip(allVectors[2], allVectors[0])])
    self.anMatrix.append([x + y + z for x, y, z in 
    zip(allVectors[0], allVectors[1], allVectors[2])])
    self.anMatrix.append([x - y - z for x, y, z in 
    zip(allVectors[0], allVectors[1], allVectors[2])])
    self.anMatrix.append([x + y - z for x, y, z in 
    zip(allVectors[0], allVectors[1], allVectors[2])])
    self.anMatrix.append([x - y + z for x, y, z in 
    zip(allVectors[0], allVectors[1], allVectors[2])])
    
    self.bnMatrix = []
    self.bnMatrix.append(bVectors[0])
    self.bnMatrix.append(bVectors[1])
    self.bnMatrix.append(bVectors[2])
    self.bnMatrix.append([x + y for x, y in zip(bVectors[0], bVectors[1])])
    self.bnMatrix.append([x - y for x, y in zip(bVectors[0], bVectors[1])])
    self.bnMatrix.append([x + y for x, y in zip(bVectors[1], bVectors[2])])
    self.bnMatrix.append([x - y for x, y in zip(bVectors[1], bVectors[2])])
    self.bnMatrix.append([x + y for x, y in zip(bVectors[2], bVectors[0])])
    self.bnMatrix.append([x - y for x, y in zip(bVectors[2], bVectors[0])])
    self.bnMatrix.append([x + y + z for x, y, z in 
    zip(bVectors[0], bVectors[1], bVectors[2])])
    self.bnMatrix.append([x - y - z for x, y, z in 
    zip(bVectors[0], bVectors[1], bVectors[2])])
    self.bnMatrix.append([x + y - z for x, y, z in 
    zip(bVectors[0], bVectors[1], bVectors[2])])
    self.bnMatrix.append([x - y + z for x, y, z in 
    zip(bVectors[0], bVectors[1], bVectors[2])])
         
    self.an2h = []
    self.bn2h = []
    for index in range(len(self.anMatrix)):
      self.an2h.append(0.5 * self.norm(self.anMatrix[index]))
      self.bn2h.append(0.5 * self.norm(self.bnMatrix[index]))
              
  '''
  Computes the norm of the three dimensional molecule.
  The norm roughly describes the length or size of the vector.
  '''
  def norm(self, a):
    return a[0]*a[0] + a[1]*a[1] + a[2]*a[2]
    
  
  def scalarProduct(self, a, b):
    return sum([elem1 * elem2 for elem1, elem2 in zip(a, b)])
    
  def crossProduct(self, a, b):
    # a.y * b.z - a.z * b.y
    firstComponent = a[1] * b[2] - a[2] * b[1]
    # a.z * b.x - a.x * b.z ,
    secondComponent = a[2] * b[0] - a[0] * b[2]
    # a.x * b.y - a.y * b.x
    thirdComponent = a[0] * b[1] - a[1] * b[0]
    return [ firstComponent, secondComponent, thirdComponent ]  
  
  
  def fold_in_ws(self, v):
    epsilon = 1.0e-10
    done = False
    maxIter = 10
    iterations = 0
    
    while not done and iterations < maxIter:
      done = True
      i = 0
      while i < 13 and done:
        sp = self.scalarProduct(v, self.anMatrix[i])
        if sp > self.an2h[i] + epsilon:
          done = False
          for index in range(len(self.anMatrix[i])):
            v[index] = v[index] - self.anMatrix[i][index]
          #v -= self.anMatrix[i]
          while self.scalarProduct(v, self.anMatrix[i]) > self.an2h[i] + epsilon :
            for index in range(len(self.anMatrix[i])):
              v[index] = v[index] - self.anMatrix[i][index]
        elif sp < -self.an2h[i] - epsilon:
          done = False
          for index in range(len(self.anMatrix[i])):
            v[index] = v[index] + self.anMatrix[i][index]
          #v -= self.anMatrix[i]
          while self.scalarProduct(v, self.anMatrix[i]) < -self.an2h[i] - epsilon :
            for index in range(len(self.anMatrix[i])):
              v[index] = v[index] + self.anMatrix[i][index]
        i += 1
      
      iterations += 1
    return v
    
  def getVolume(self):
    return self.volume
   
##############################################################
'''
Filename: gofr.py
'''
import unitcell
import sys
from math import sqrt, pi, pow
import time
import pprint
import ast
import csv
import gofr_config as gc
import collections

class Gofr(object):
  def __init__(self, xyzInput, name1, name2, rmax, dr, fileName):
    self.xyzInput = xyzInput
    self.name1 = name1
    self.name2 = name2
    self.rmax = rmax
    self.dr = dr
    self.fileName = fileName
    
  def process(self):
    name1 = self.name1
    name2 = self.name2
    rmax = self.rmax
    dr = self.dr
    nconfig = 0
    # This list holds the position of the atoms
    line_index = 0
    # Store the file contents in the list
    file_content = self.xyzInput
    # Read the number of atoms in the first line
    number_atoms = int(file_content[line_index])
    # Corresponds to the first atom
    isp1 = [0] * number_atoms
    # Corresponds to the second atom
    isp2 = [0] * number_atoms
    tau = [None] * number_atoms
    names_of_atoms = [None] * number_atoms
    species_1_count = 0
    species_2_count = 0
    line_index += 1
    nbin = (int) (rmax / dr)
    count = [0.0] * nbin
    omega = 0.0
    cum_counts = {}
    # Loop through the file contents line by line
    while line_index < len(file_content):
      nconfig += 1
    
      line_clean = file_content[line_index].split()

      # The iteration number
      iteration = line_clean[0]
      # The D3 vectors
      a0 = [float(x) for x in line_clean[1:4]]
      a1 = [float(x) for x in line_clean[4:7]]
      a2 = [float(x) for x in line_clean[7:10]]

      line_index += 1

      uc = unitcell.UnitCell(a0, a1, a2)
      omega = uc.getVolume()

      for i in range(number_atoms):
        line_clean = file_content[line_index].split()
        line_index += 1
      
        nm = line_clean[0]
        d3_vec = [float(x) for x in line_clean[1:4]]
        tau[i] = d3_vec
        if nconfig == 1:
          names_of_atoms.append(nm)
          if name1 in nm:
            isp1[species_1_count] = i
            species_1_count += 1
          if name2 in nm:
            isp2[species_2_count] = i
            species_2_count += 1
      
          
      for i in range(species_1_count):
        for j in range(species_2_count):
          first_vector = tau[isp1[i]]
          second_vector = tau[isp2[j]]
          
          difference_vector = [ai - bi for ai, bi in 
          zip(first_vector, second_vector)]
          difference_vector = uc.fold_in_ws(difference_vector)
          scaled_vec = [x / dr for x in difference_vector]
          
          length_scaled_vec = sqrt(scaled_vec[0] ** 2 + scaled_vec[1] ** 2 
          + scaled_vec[2] ** 2)
         
          k = (int) (length_scaled_vec + 0.5)
          if k < nbin:
            count[k] += 1
      
      #print count
      if nconfig % gc.stepsize == 0:
        cum_counts[nconfig] = ast.literal_eval(str(count))

      #print '\n\n'
      #print cum_counts
       
      if line_index < len(file_content):
        number_atoms = int(file_content[line_index])
        line_index += 1
    
    if species_1_count == 0:
      print ' no atoms %s found.' % name1
      sys.exit(1)
    if species_2_count == 0:
      print ' no atoms %s found.' % name2
      sys.exit(1)
  
    print '%d %s atoms found.' % (species_1_count, name1)
    print '%d %s atoms found.' % (species_2_count, name2)
  
        
    # normalization differs for same species vs different species
    npairs = 0
    if name1 == name2:
      npairs = species_1_count * (species_2_count - 1)
    else:
      npairs = species_1_count * species_2_count
    
    visual_data = []
    initial_data = ['NConfig', 'Radius', 'G(r)']
    visual_data.append(initial_data)
    counter = 1

    cum_counts = collections.OrderedDict(sorted(cum_counts.items()))

    for key, val in cum_counts.items():
      for i in range(1, nbin):
        r = i * dr
        rmin = (i - 0.5) * dr
        rmax = (i + 0.5) * dr
        vshell = (4.0 * pi / 3.0) * (pow(rmax, 3.0) - pow(rmin, 3.0))
        count_id = (vshell * npairs) / omega
        g = val[i] / (count_id * gc.stepsize)
        temp_data = [counter, r, g]
        visual_data.append(temp_data )
      
      # Counter in outer loop records which config number it is
      counter += 1
           
    with open(self.fileName,'w+') as fp:
        a = csv.writer(fp, delimiter=',')
        a.writerows(visual_data)
    
    n = 0.0
    for i in range(1, nbin):
      r = i * dr
      n += count[i] / (species_1_count * nconfig)
      print 'r, n = %s, %s' % (r, n)
    

##############################################################
'''
Filename: visualizer.py
'''


#!/usr/bin/python
import pprint
import gofr_config as gc
import urllib2
import sys
import pprint
import os.path
import getopt
import csv

def handle_files(fp, source_file):
  fp = None
  # test if input_source is a local file
  # if not, process as a URL
  print 'sf = ', source_file
  if ( os.path.isfile(source_file) ):
    fp = open(source_file)
  else:
    # attempt to open as a URL
    try:
      fp = urllib2.urlopen(source_file)
    except (ValueError,urllib2.HTTPError) as e:
      print 'Error opening the URL'
      #sys.exit()
  
  return fp 


argc = len(sys.argv)
input_source1 = sys.argv[1]
input_source2 = ''

# ./visualizer.py cum_counts.txt 
# ./visualizer.py cum_counts.txt -x 5 -y 5
# ./visualizer.py cum_counts.txt cum_counts2.txt
# ./visualizer.py cum_counts.txt cum_counts2.txt -x 5 -y 5
if argc > 2:
  if sys.argv[2] == '':
    input_source2 = ''
  elif sys.argv[2] == '-x' or sys.argv[2] == '--x':
    input_source2 = ''
  # If it is a file
  elif ".txt" in sys.argv[2]:
    input_source2 = sys.argv[2]

try:
  if input_source2 == '':
    opts, args = getopt.getopt(sys.argv[2:], "hx:y:", ["help", "xlim=", "ylim="])
  else:
    opts, args = getopt.getopt(sys.argv[3:], "hx:y:", ["help", "xlim=", "ylim="])
except getopt.GetoptError:
  print 'visualizer.py counts.dat -x 4 -y 4'
  sys.exit(2)
  
#print opts
#print args
xlim = ''
ylim = ''
for o, a in opts:
    if o in ('-x', '--xlim'):
        xlim=a
    elif o in ('-y', '--ylim'):
        ylim=a
    else:
        print("Usage: %s -x 4 -y 4" % sys.argv[0])

fp = None
fp = handle_files(fp, input_source1)

vis_data = []
first_source_vals = {}
second_source_vals = {}

if input_source2 == '':
  vis_data.append(['NConfig','Radius', 'Gr1']) # Only one file
elif input_source2 != '':
  vis_data.append(['NConfig','Radius', 'Gr1', 'Gr2']) # Two files

rownum = 0
try:
  reader = csv.reader(fp) 
  for row in reader:
    if rownum != 0:
      key = row[0] + ',' + row[1]
      first_source_vals[key] = row[2]
      float_list = [float(x) for x in row]
      if input_source2 == '':
        vis_data.append(float_list)
    rownum += 1
finally:
  fp.close()
  
if input_source2 != '':
  fp = handle_files(fp, input_source2)
  rownum = 0
  try:
    reader = csv.reader(fp) 
    for row in reader:
      if rownum != 0:
        try:
          key = row[0] + ',' + row[1]
          tempList = [float(row[0]), float(row[1]), 
          float(first_source_vals[key]), float(row[2]) ]
          vis_data.append(tempList)
        except KeyError:
          continue
      rownum += 1
  finally:
    fp.close()

def seq(start, stop, step=1):
    n = int(round((stop - start)/float(step)))
    if n > 1:
        return([start + step*i for i in range(n+1)])
    else:
        return([])
  
#pprint.pprint(vis_data)

class visualizer(object):

  def __init__(self, data, nconfig, xlim, ylim, twoInputFiles):
    self.data = data
    self.total_nconfigs = nconfig
    self.xlim = xlim
    self.ylim = ylim
    self.twoInputFiles = twoInputFiles
    
  def generateVisualFile(self):
    tickValues = ''
    if self.xlim == '':
      tickValues = str(seq(0, float(self.data[-1][1]), 0.5))
    else:
      tickValues = str(seq(0, float(self.xlim), 0.5))

    columnStringInterpolation = ''
    gr2Add = ''
    dataframeAdjust = ''

    if not self.twoInputFiles:
      columnStringInterpolation = "'columns': [1, 2]"
      gr2Add = "//columnsTable.addColumn('number', 'Gr2');"
      dataframeAdjust = r'''
      columnsTable.addRow([leftValue, data.getValue(left_nconfigs[i], 1),
      (right_gofr_1-left_gofr_1)/(rightValue - leftValue)]);
      '''
    else:
      columnStringInterpolation = "'columns': [1, 2, 3]"
      gr2Add = "columnsTable.addColumn('number', 'Gr2');"
      dataframeAdjust = r'''
      var right_gofr_2 = data.getValue(right_nconfigs[i], 3);
      var left_gofr_2 =  data.getValue(left_nconfigs[i], 3); 
      columnsTable.addRow([leftValue, data.getValue(left_nconfigs[i], 1),
      (right_gofr_1-left_gofr_1)/(rightValue - leftValue),(right_gofr_2-left_gofr_2)
      /(rightValue - leftValue)]);
      '''

    xAxisString = ''
    yAxisString = ''
    if self.xlim != '':
      xAxisString = r'''
      hAxis: {viewWindow: {max: %s}, title: 'r (Angstrom)', ticks: %s},
      ''' % (self.xlim, tickValues)
    if self.ylim != '':
      yAxisString = r'''
      vAxis: {viewWindow: {max: %s}, title: 'g(r)'}
      ''' % self.ylim
  
    if xAxisString == '' or yAxisString == '':
      xAxisString = "hAxis: {title: 'r (Angstrom)', ticks: %s},\n" % tickValues
      yAxisString = "vAxis: {title: 'g(r)'}\n" 

    boundsString = xAxisString + yAxisString

    htmlString = r"""
    <html>
      <head>
        <!--Load the AJAX API-->
        <script type="text/javascript" src="https://www.gstatic.com/charts/
        loader.js"></script>
        <script type="text/javascript">

          // Load the Visualization API and the controls package.
          google.charts.load('current', {'packages':['corechart', 'controls', 
          'table']});

          // Set a callback to run when the Google Visualization API is loaded.
          google.charts.setOnLoadCallback(drawDashboard);

          // Callback that creates and populates a data table,
          // instantiates a dashboard, a range slider and a pie chart,
          // passes in the data and draws it.
          function drawDashboard() {

            // Create our data table.
            var data = google.visualization.arrayToDataTable(%s);

            // Create a dashboard.
            var dashboard = new google.visualization.Dashboard(document.
            getElementById('dashboard_div'));

            var imageLink = document.getElementById('image_link');
            // Create a range slider, passing some options
            var nConfigSlider = new google.visualization.ControlWrapper({
              'controlType': 'NumberRangeFilter',
              'containerId': 'filter_div',
              'options': {
                'filterColumnLabel': 'NConfig',
                "ui": {"label": "Frame Number"}
              }
             
            });
            
            var myLine = new google.visualization.ChartWrapper({
                'chartType' : 'LineChart',
                'containerId' : 'chart_div',
                'view': {%s},

                'options': {
                  'title': 'g_%s%s(r)',
                  width: 1200, 
                  height: 800, 
                  pointSize: 5, 
                  lineWidth: 2,
                  %s
                },
                'dataTable': data
            });
           
            var myTable = new google.visualization.ChartWrapper({
               'chartType' : 'Table',
               'containerId' : 'table_div',
               'width': 800,
               'height': 600,
             });
             
            
            dashboard.bind(nConfigSlider, myTable);
            // Establish dependencies, declaring that 'filter' drives 'pieChart',
            // so that the pie chart will only display entries that are let through
            // given the chosen slider range.
            dashboard.bind(nConfigSlider, myLine);

            // Draw the dashboard.
            dashboard.draw(data);
            
            google.visualization.events.addListener(
            nConfigSlider, 'ready', setInitialState);
            google.visualization.events.addListener(
            nConfigSlider, 'statechange', setInitialState);
            
            function setInitialState() {
              // Temporary Filtered Table
              var columnsTable = new google.visualization.DataTable();
              columnsTable.addColumn('number', 'nconfig');
              columnsTable.addColumn('number', 'Radius');
              columnsTable.addColumn('number', 'Gr1');
              %s
              
              // Get Slider Information
              var sliderState = nConfigSlider.getState();
              var leftValue = sliderState.lowValue;
              var rightValue = sliderState.highValue;
              
            
              // Corresponds to indices in original data Ex:- [5, 6, 7, 8, 9]
              var left_nconfigs = data.getFilteredRows(
              [{column : 0, value: leftValue}]);
              var right_nconfigs = data.getFilteredRows(
              [{column : 0, value: rightValue}]);
              
              for(var i = 0; i < left_nconfigs.length; i++) {
                var right_gofr_1 = data.getValue(right_nconfigs[i], 2);
                var left_gofr_1 =  data.getValue(left_nconfigs[i], 2); 
                
                %s
              }
              myLine.setDataTable(columnsTable);
              myLine.draw();
              
              imageLink.innerHTML = '<img src="' + myLine.getChart().getImageURI() 
              + '">';
              console.log(imageLink.innerHTML);
            }
            
          }
        </script>
      </head>

      <body class="container">
        <!--Div that will hold the dashboard-->
        <div id="dashboard_div">
          <div id="info">
            NConfigs: %s <br /> 
            Radius: %s <br /> 
            Dr: %s <br />
            First Molecule Name: %s <br />
            Second Molecule Name: %s <br />
          </div>
          <!--Divs that will hold each control and chart-->
          <div id="filter_div"></div>
          <div id="chart_div"></div>
          <div id="display_div"></div>
          <div id="image_link"></div>
          <div id="table_div"><!-- Table renders here --></div>
          
          
        </div>
      </body>
    </html>
    """ % (
      pprint.pformat(self.data), 
      columnStringInterpolation,
      gc.first_molecule_name, 
      gc.second_molecule_name,
      boundsString,
      gr2Add,
      dataframeAdjust,
      str(self.total_nconfigs) , 
      str(gc.rmax), 
      str(gc.dr), 
      str(gc.first_molecule_name), 
      str(gc.second_molecule_name)
      )
    # CURRENTLY X AND Y LIMITS ARE TAKEN OUT
    #final_filename = "%s_%s_stepsize%s.html" % 
    #(gc.first_input_source, gc.second_input_source, gc.stepsize)
    final_filename = "test1.html"
    f = open(final_filename,'w+')
    f.write(htmlString)

nconfig = vis_data[-1][0]
twoFiles = False
if input_source2 != '':
  twoFiles = True
vis_obj = visualizer(vis_data, nconfig, xlim, ylim, twoFiles) 
vis_obj.generateVisualFile()
print ('Done generating visual file!')


\end{minted}